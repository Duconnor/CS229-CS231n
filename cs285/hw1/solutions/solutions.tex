\documentclass[10pt, a4paper]{ctexart}
\usepackage[margin=1in]{geometry}
\usepackage{bm}
\usepackage{amsmath}
\usepackage{graphicx}
\usepackage{float}
\usepackage{listings}

\renewcommand{\figurename}{Figure.}
\renewcommand{\tablename}{Table.}

\begin{document}
\title{Assignment 1: Imitation Learning}
\date{}
\author{}
\maketitle

\section{Behavioral Cloning}
{\bf{2.}}
\begin{table}[H]
    \centering
    \begin{tabular}{c|c|c|c|c}
        \hline
        \hline
        Tasks&Eval Mean&Eval Std&Train Mean&Train Std\\
        \hline
        Ant&2496&235.8&4714&12.2\\
        \hline
        Humanoid&264.4&49.51&10345&20.98\\
        \hline
    \end{tabular}
    \caption{Performance of Behavioral Cloning on Different Tasks}
\end{table}
{\bf{3.}}
\begin{figure}[H]
    \centering
    \includegraphics[width=0.7\linewidth]{../plot/mean.png}
    \caption{Mean}
\end{figure}
I choose the number of training steps per iteration because it is an important hyperparamter, which can significantly affect the performance.

\section{DAgger}
{\bf{2.}}
\begin{figure}[H]
    \centering
    \includegraphics[width=0.7\linewidth]{../plot/dagger.png}
    \caption{Performance of DAgger on Ant-v2 (Used Default Setting with Eval Batch Size=5000)}
\end{figure}
\end{document}